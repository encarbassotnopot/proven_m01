\documentclass[a4paper,12pt]{article}
\usepackage[utf8]{inputenc}

\usepackage{fancyhdr} % Per la capçalera
\usepackage{geometry}
\usepackage{graphicx} % Per importar logos (i altres gràfics)
\usepackage[colorlinks,linkcolor=black]{hyperref} % Per fer que l'index tingui hiperlinks en el pdf
\usepackage{parskip}
\usepackage[sc,small]{titlesec} % Seccions personalitzades

%% Entreliniats
\linespread{1.5}

%% Capçalera
\pagestyle{fancy}
\setlength{\headheight}{40pt}
\addtolength{\topmargin}{-20pt}
\fancyhead[L]{\includegraphics*[height=\headheight]{provencana_bw.pdf}}
\fancyhead[R]{
	{\scshape\scriptsize Mòdul 1: Sistemes Informàtics}\\
	{\scshape\footnotesize Activitat - Codificació de Caràcters}\\
	{\scshape\small Eina Coma Bages}}

\addtolength{\textheight}{2cm}

%% Comandes personalitzades
\newcommand{\mygraphic}[2][width=0.9\textwidth]{\begin{center}
		\centering\includegraphics[#1]{#2}\par
\end{center}}

\renewcommand{\contentsname}{Índex}

% Format enunciat
\newcounter{activitatnumero}
\setcounter{activitatnumero}{0}
\newcommand{\enunciat}[1]{
	\stepcounter{activitatnumero}
	{\large\arabic{activitatnumero}}. #1
	\par
}

\begin{document}
\begin{titlepage}
	\centering
	\includegraphics*[width=0.15\textwidth]{provencana_color.pdf}
	\par\vspace{0.5cm}

	{\scshape\Large Institut Provençana \par}

	\vspace{1cm}
	
	{\itshape\Large Activitat \par}
	{\bfseries\LARGE Codificació de Caràcters \par}
	
	\vspace{1cm}

	{\scshape\large Mòdul 1: \par}
	{\scshape\Large Sistemes Informàtics \par}

	\vspace{0.5cm}
	
	{\scshape\normalsize Unitat Formativa 1: \par}
	{\scshape\large Instal·lació, Configuració \\ i Explotació del Sistema Informàtic\par}

	\vfill
	{\Large\itshape Eina Coma Bages\par}
	
	\vfill
	Curs 2022/2023
\end{titlepage}

% \tableofcontents


\enunciat{A continuació es dona la posició de 6 caràcters corresponents a la taula ASCII reduïda. Amb l'ajuda de les tecles “ALT + Teclat Numèric” troba els caràcters corresponents. Emplena la resta de columnes de la taula.}
\enunciat{A continuació es dona la posició de 6 caràcters corresponents a la taula ASCII reduïda. Amb l'ajuda de les tecles “ALT + Teclat Numèric” troba els caràcters corresponents. Emplena la resta de columnes de la taula.}

\end{document}