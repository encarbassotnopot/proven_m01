\documentclass[a4paper,12pt]{report}
\usepackage[utf8]{inputenc}

\usepackage{array} % Per ajuntar cel·les de la taula
\usepackage{calc}
\usepackage{changepage} % Per poder canviar l'amplada del text en la resposta
\usepackage{fancyhdr} % Per la capçalera
\usepackage{geometry}
\usepackage{graphicx} % Per importar logos (i altres gràfics)
\usepackage{mathtools}

\newenvironment{resposta}
	{\begin{adjustwidth}{3em}{2em}}
	{\end{adjustwidth}}

%% Entreliniats
\setlength{\jot}{5pt}
\linespread{1.5}
\setlength{\parindent}{0em}

%% Capçalera
\pagestyle{fancy}
\setlength{\headheight}{40pt}
\addtolength{\topmargin}{-20pt}
\fancyhead[L]{\includegraphics*[height=\headheight]{provencana_bw.pdf}}
\fancyhead[R]{
	{\scshape\scriptsize Mòdul 1: Sistemes Informàtics}\\
	{\scshape\footnotesize Act 1 - Sistemes de Numeració}\\
	{\scshape\small Eina Coma Bages}}

\begin{document}
\begin{titlepage}
	\centering
	\includegraphics*[width=0.15\textwidth]{provencana_color.pdf}
	\par\vspace{0.5cm}

	{\scshape\Large Institut Provençana \par}

	\vspace{1cm}
	
	{\itshape\Large Activitat 1 \par}
	{\bfseries\LARGE Sistemes de Numeració \par}
	
	\vspace{1cm}

	{\scshape\large Mòdul 1: \par}
	{\scshape\Large Sistemes Informàtics \par}

	\vspace{0.5cm}
	
	{\scshape\normalsize Unitat Formativa 1: \par}
	{\scshape\large Instal·lació, Configuració \\ i Explotació del Sistema Informàtic\par}

	\vfill
	{\Large\itshape Eina Coma Bages\par}
	
	\vfill
	Curs 2022/2023

\end{titlepage}

\textsc{{\large 1.}  Descompassar el número octal 3565 segons el Teorema Fonamental de la Numeració:}
\vspace{-5pt}
\begin{gather*}
	3565_8 = 3\cdot8^3 + 5\cdot8^2 + 6\cdot8^1 + 5\cdot8^0 = \\
	= 3\cdot(2^3)^3 + 5\cdot(2^3)^2 + 6\cdot(2^3)^1 + 5\cdot(2^3)^0 = \\
	= 3\cdot2^9 + 5\cdot2^6 + 6\cdot2^3 + 5\cdot2^0 = \\
	= 3\cdot512 + 5\cdot64 + 6\cdot8 + 5\cdot1 = \\
	= 1536 + 320 + 485 + 5 = 1909 
\end{gather*}

\textsc{{\large 2.} Quants números binaris diferents podem codificar amb 5 bits?}

\begin{resposta}
	Amb 5 bits disposem de 5 dígits que poden prendre dos valors (0 o 1) cada un, per tant ens tenim $2^5$ o 32 combinacions diferents.
\end{resposta}
\vspace{1em}

\textsc{{\large 3.}  Quin és el número més gran que podem representar amb 6 bits?}

\begin{resposta}
	Amb 6 bits el nombre més gran que podem expressar és 111111. De la mateixa manera que amb 5 bits, teníem $2^5$ nombres diferents, amb 6 bits en tenim $2^6$ o 64. Com que comencem a comptar des de 0, això vol dir que el més gran és 63.

	Ho podem comprovar convertint-lo a decimal:
	$2^5+2^4+2^3+2^2+2^1+2^0 = 32+16+8+4+2+1 = 63$
\end{resposta}

\pagebreak

\textsc{{\large 4.}  Resoldre les següents conversions}

\begin{resposta}
	La conversió entre octal i binari i hexadecimal i binari és senzilla, ja que les bases 8 i 16 són potències de 2. Cada dígit d'octal es pot substituir directament per 3 de binari ($8 = 2^3$), i cada xifra d'hexadecimal per 4 ($16 = 2^4$).
\end{resposta}

\begin{center}
	\begin{tabular}{c c c c c}
		\cline{1-2} \cline{4-5}
		Oct i Hex & Binari && Hex & Binari \\
		\cline{1-2} \cline{4-5}
		0 & 0000 & & 8 & 1000 \\
		\cline{1-2} \cline{4-5}
		1 & 0001 & & 9 & 1001 \\
		\cline{1-2} \cline{4-5}
		2 & 0010 & & A & 1010 \\
		\cline{1-2} \cline{4-5}
		3 & 0011 & & B & 1011 \\
		\cline{1-2} \cline{4-5}
		4 & 0100 & & C & 1100 \\
		\cline{1-2} \cline{4-5}
		5 & 0101 & & D & 1101 \\
		\cline{1-2} \cline{4-5}
		6 & 0110 & & E & 1110 \\
		\cline{1-2} \cline{4-5}
		7 & 0111 & & F & 1111 \\
		\cline{1-2} \cline{4-5}
	\end{tabular}
\end{center}

\begin{resposta}
	Per fer la conversió des de decimal o cap a decimal no tenim més remei que calcular-la.
\end{resposta}

\vspace{1em}
\begin{center}
		\begin{tabular}{
			!{\vrule width 1.4pt} *{3}{c|} c
			!{\vrule width 1.4pt} *{3}{c|} c
			!{\vrule width 1.4pt} *{2}{c|} c
			!{\vrule width 1.4pt} *{9}{c|} c
			!{\vrule width 1.4pt}}
			\noalign{\hrule height 1.4pt}
			\multicolumn{4}{!{\vrule width 1.4pt}c!{\vrule width 1.4pt}}{Decimal} &
			\multicolumn{4}{c!{\vrule width 1.4pt}}{Hexadecimal} &
			\multicolumn{3}{c!{\vrule width 1.4pt}}{Octal} &
			\multicolumn{10}{c!{\vrule width 1.4pt}}{Binari} \\
			\noalign{\hrule height 1.4pt}
			0 & 2 & 0 & 0 & 0 & 0 & C & 8 & 3 & 1 & 0 & 0 & 0 & 1 & 1 & 0 & 0 & 1 & 0 & 0 & 0 \\
			\noalign{\hrule}
			0 & 2 & 2 & 3 & 0 & 0 & D & F & 3 & 3 & 7 & 0 & 0 & 1 & 1 & 0 & 1 & 1 & 1 & 1 & 1 \\
			\noalign{\hrule}
			0 & 0 & 8 & 7 & 0 & 0 & 5 & 7 & 1 & 2 & 7 & 0 & 0 & 0 & 1 & 0 & 1 & 0 & 1 & 1 & 1 \\
			\noalign{\hrule}
			0 & 1 & 6 & 7 & 0 & 0 & A & 7 & 2 & 4 & 7 & 0 & 0 & 1 & 0 & 1 & 0 & 0 & 1 & 1 & 1 \\
			\noalign{\hrule height 1.4pt}
		\end{tabular}
\end{center}
\begin{resposta}
	\setlength{\jot}{-5pt}
	$200_{10}$: \vspace{-1em}
	\begin{align*}
		200/16 &= 12, \textnormal{ residu } 8; \\
		12/16 &= 0, \textnormal{ residu } 12 \textnormal{ (C, en hexadecimal)} \\
		200_{10} &= C8_{16}
	\end{align*}

	\vspace{-0.5em}

	$DF_{16}$: $13\cdot16+15=208+15=223_{10}$ \\
	$57_{16}$: $5\cdot16+7=80+7=87_{10}$ \\
	$A7_{16}$: $10\cdot16+7=160+7=167_{10}$
\end{resposta}

\vspace{1em}
\begin{center}
		\begin{tabular}{
			!{\vrule width 1.4pt} *{3}{c|} c
			!{\vrule width 1.4pt} *{3}{c|} c
			!{\vrule width 1.4pt} *{2}{c|} c
			!{\vrule width 1.4pt} *{9}{c|} c
			!{\vrule width 1.4pt}}
			\noalign{\hrule height 1.4pt}
			\multicolumn{4}{!{\vrule width 1.4pt}c!{\vrule width 1.4pt}}{Decimal} &
			\multicolumn{4}{c!{\vrule width 1.4pt}}{Hexadecimal} &
			\multicolumn{3}{c!{\vrule width 1.4pt}}{Octal} &
			\multicolumn{10}{c!{\vrule width 1.4pt}}{Binari} \\
			\noalign{\hrule height 1.4pt}
			0 & 0 & 6 & 0 & 0 & 0 & 3 & C & 0 & 7 & 4 & 0 & 0 & 0 & 0 & 1 & 1 & 1 & 1 & 0 & 0 \\
			\noalign{\hrule}
			0 & 0 & 9 & 1 & 0 & 0 & 5 & B & 1 & 3 & 3 & 0 & 0 & 0 & 1 & 0 & 1 & 1 & 0 & 1 & 1 \\
			\noalign{\hrule}
			0 & 0 & 4 & 8 & 0 & 0 & 3 & 0 & 0 & 6 & 0 & 0 & 0 & 0 & 0 & 1 & 1 & 0 & 0 & 0 & 0 \\
			\noalign{\hrule}
			0 & 0 & 6 & 3 & 0 & 0 & 3 & F & 0 & 7 & 7 & 0 & 0 & 0 & 0 & 1 & 1 & 1 & 1 & 1 & 1 \\
			\noalign{\hrule height 1.4pt}
	\end{tabular}
\end{center}
\begin{resposta}
	\setlength{\jot}{-5pt}
	$60_{10}$: \vspace{-1em}
	\begin{align*}
		60/16 &= 3, \textnormal{ residu } 12 \textnormal{ (C, en hexadecimal)}; \\
		3/16 &= 0, \textnormal{ residu } 3 \\
		60_{10} &= 3C_{16}
	\end{align*}

	\vspace{-0.5em}

	$5B_{16}$: $5\cdot16+11=80+11=91_{10}$ \\
	$30_{16}$: $3\cdot16+0=48_{10}$ \\
	$3F_{16}$: $3\cdot16+15=48+15=63_{10}$
\end{resposta}

\vspace{1em}
\begin{center}
	\begin{tabular}{
		!{\vrule width 1.4pt} *{3}{c|} c
		!{\vrule width 1.4pt} *{3}{c|} c
		!{\vrule width 1.4pt} *{2}{c|} c
		!{\vrule width 1.4pt} *{9}{c|} c
		!{\vrule width 1.4pt}}
		\noalign{\hrule height 1.4pt}
		\multicolumn{4}{!{\vrule width 1.4pt}c!{\vrule width 1.4pt}}{Decimal} &
		\multicolumn{4}{c!{\vrule width 1.4pt}}{Hexadecimal} &
		\multicolumn{3}{c!{\vrule width 1.4pt}}{Octal} &
		\multicolumn{10}{c!{\vrule width 1.4pt}}{Binari} \\
		\noalign{\hrule height 1.4pt}0 & 1 & 9 & 2 & 0 & 0 & C & 0 & 3 & 0 & 0 & 0 & 0 & 1 & 1 & 0 & 0 & 0 & 0 & 0 & 0\\
		\noalign{\hrule}
		0 & 1 & 6 & 8 & 0 & 0 & A & 8 & 2 & 5 & 0 & 0 & 0 & 1 & 0 & 1 & 0 & 1 & 0 & 0 & 0\\
		\noalign{\hrule}
		  &   &   &   &   &   &   &   & 1 & 9 & 2 &   &   &   &   &   &   &   &   &   &  \\
		\noalign{\hrule}
		0 & 2 & 1 & 7 & 0 & 0 & D & 9 & 3 & 3 & 1 & 0 & 0 & 1 & 1 & 0 & 1 & 1 & 0 & 0 & 1\\
		\noalign{\hrule height 1.4pt}
\end{tabular}
\end{center}
\begin{resposta}
	\setlength{\jot}{-5pt}
	$192_{10}$: \vspace{-1em}
	\begin{align*}
		192/16 &= 12, \textnormal{ residu } 0; \\
		12/16 &= 0, \textnormal{ residu } 12 \textnormal{ (C, en hexadecimal)} \\
		192_{10} &= C0_{16}
	\end{align*}

	\vspace{-0.5em}

	$A8_{16}$: $10\cdot16+8=160+8=168_{10}$ \\
	La tercera fila no es pot resoldre, ja que el número 9 no existeix en octal.\\
	$D9_{16}$: $13\cdot16+9=208+9=217_{10}$
\end{resposta}

\vspace{1em}
\begin{center}
	\begin{tabular}{
		!{\vrule width 1.4pt} *{3}{c|} c
		!{\vrule width 1.4pt} *{3}{c|} c
		!{\vrule width 1.4pt} *{2}{c|} c
		!{\vrule width 1.4pt} *{9}{c|} c
		!{\vrule width 1.4pt}}
		\noalign{\hrule height 1.4pt}
		\multicolumn{4}{!{\vrule width 1.4pt}c!{\vrule width 1.4pt}}{Decimal} &
		\multicolumn{4}{c!{\vrule width 1.4pt}}{Hexadecimal} &
		\multicolumn{3}{c!{\vrule width 1.4pt}}{Octal} &
		\multicolumn{10}{c!{\vrule width 1.4pt}}{Binari} \\
		\noalign{\hrule height 1.4pt}
		0 & 2 & 1 & 0 & 0 & 0 & D & 2 & 3 & 2 & 2 & 0 & 0 & 1 & 1 & 0 & 1 & 0 & 0 & 1 & 0\\
		\noalign{\hrule}
		0 & 2 & 1 & 3 & 0 & 0 & D & 5 & 3 & 2 & 5 & 0 & 0 & 1 & 1 & 0 & 1 & 0 & 1 & 0 & 1\\
		\noalign{\hrule}
		0 & 0 & 7 & 3 & 0 & 0 & 4 & 9 & 1 & 1 & 1 & 0 & 0 & 0 & 1 & 0 & 0 & 1 & 0 & 0 & 1\\
		\noalign{\hrule}
		0 & 1 & 6 & 9 & 0 & 0 & A & 9 & 2 & 5 & 1 & 0 & 0 & 1 & 0 & 1 & 0 & 1 & 0 & 0 & 1\\
		\noalign{\hrule height 1.4pt}
\end{tabular}
\end{center}
\begin{resposta}
	\setlength{\jot}{-5pt}
	$210_{10}$: \vspace{-1em}
	\begin{align*}
		210/16 &= 13, \textnormal{ residu } 2; \\
		13/16 &= 0, \textnormal{ residu } 13 \textnormal{ (D, en hexadecimal)} \\
		210_{10} &= D2_{16}
	\end{align*}
	
	\vspace{-0.5em}

	$D5_{16}$: $13\cdot16+5=208+5=213_{10}$ \\
	$49_{16}$: $4\cdot16+9=64+9=73_{10}$ \\
	$A9_{16}$: $10\cdot16+9=160+9=169_{10}$
\end{resposta}

\end{document}