\documentclass[a4paper,12pt]{article}
\usepackage[utf8]{inputenc}

\usepackage[catalan]{babel} % Patrons de trencament de paraula
\usepackage{fancyhdr} % Per la capçalera
\usepackage{geometry}
\usepackage{graphicx} % Per importar logos (i altres gràfics)
\usepackage[colorlinks,linkcolor=black]{hyperref} % Per fer que l'index tingui hiperlinks en el pdf
\usepackage{indentfirst}
\usepackage{titlesec} % Seccions personalitzades

%% Títols
\newcommand{\modulnum}{1}
\newcommand{\modulnom}{Sistemes Informàtics}

\newcommand{\ufnum}{1}
\newcommand{\ufnom}{Instal·lació, Configuració \\ i Explotació del Sistema Informàtic}

\newcommand{\acttipus}{Pràctica 5.3}
\newcommand{\actnom}{Configuració de paràmetres de la Xarxa a Windows}

%% Entreliniats
\linespread{1.5}

%% Capçalera
\pagestyle{fancy}
\setlength{\headheight}{55pt}
\addtolength{\topmargin}{-25pt}
\fancyhead[L]{\includegraphics*[height=35pt]{provencana_bw.pdf}}
\fancyhead[R]{
	{\scshape\scriptsize Mòdul \modulnum: \modulnom}\\
	{\scshape\footnotesize \acttipus \space - \actnom}\\
	{\scshape\small Eina Coma Bages}}

\addtolength{\textheight}{2cm}

%% Secció
\titleformat{\enunciat}[runin]{\scshape}{\thesection.}{0.5em}{}


%% Comandes personalitzades
\newcommand{\mygraphic}[2][width=\textwidth]{\begin{center}
		\centering\includegraphics[#1]{#2}\par
\end{center}}

\newcounter{activitat}
\newcommand{\enunciat}[1]{\refstepcounter{activitat}
\vspace{0.5em} {\large\textit\theactivitat.} {\textit{#1}} \vspace{0.5em}}

\renewcommand{\contentsname}{Índex}

\begin{document}

\begin{titlepage}
	\centering
	\includegraphics*[width=0.15\textwidth]{provencana_color.pdf}
	\par\vspace{0.5cm}

	{\scshape\Large Institut Provençana \par}

	\vspace{1cm}
	
	{\itshape\Large Activitat \par}
	{\bfseries\LARGE Codificació de Caràcters \par}
	
	\vspace{1cm}

	{\scshape\large Mòdul 1: \par}
	{\scshape\Large Sistemes Informàtics \par}

	\vspace{0.5cm}
	
	{\scshape\normalsize Unitat Formativa 1: \par}
	{\scshape\large Instal·lació, Configuració \\ i Explotació del Sistema Informàtic\par}

	\vfill
	{\Large\itshape Eina Coma Bages\par}
	
	\vfill
	Curs 2022/2023
\end{titlepage}
\newpage

\tableofcontents
\newpage

\section{ipconfig}
Mostrem tota la informació  de les interficies de xarxa amb \texttt{ipconfig /all}.
\mygraphic{imatges/1.png}

\newpage
\section{ping}
Fem ping al DNS de Google \texttt{ping 8.8.8.8} fins que interrompem el programa (\texttt{-t}) i esperant com a màxim 500 ms (\texttt{-w 500}).
\mygraphic{imatges/2.png}

\newpage
\section{arp}
Mostrem la taula de resolució d'adreces MAC guardades a la memòria en cau \texttt{arp -a}.

Esborrem una entrada (\texttt{arp -d 255.255.255.255}) i la tornem a crear (\texttt{arp -s 255.255.255.255 ff-ff-ff-ff-ff-ff}).
\mygraphic{imatges/3.png}

\newpage
\section{tracert}
Mirem la ruta que seguim per arribar al servidor DNS de Google \texttt{tracert 8.8.8.8}.
\mygraphic{imatges/4.png}

\newpage
\section{route}
Mostrem la taula d'enrutament amb \texttt{route PRINT}.
\mygraphic{imatges/5.png}

\newpage
\section{netstat}
Mostrem les estadístiques del protocol IPv4 amb \texttt{netstat -sp ip} i després les de la interfície d'ethernet amb \texttt{netstat -e}.
\mygraphic{imatges/6.png}

\newpage
\section{nbtstat}
Amb \texttt{nbtstat -A 10.0.2.15} imprimim la taula de noms dels ordinadors connectats al nostre mitjançant la nostra direcció IP.
\mygraphic{imatges/7.png}

\newpage
\section{nslookup}
Iniciem nslookup i hi entrem \texttt{google.com}. Per defecte es mostra només el registre A (adreça IPv4), però si introduïm l'ordre \texttt{set type=ANY} els podem veure tots (com l'AAAA per l'adreça IPv6, el registre MX per enviar correus o el NS per definir \textit{authoritative name servers}).
\mygraphic{imatges/8.png}

\end{document}