\documentclass[a4paper,12pt]{article}
\usepackage[utf8]{inputenc}

\usepackage[catalan]{babel} % Patrons de trencament de paraula
\usepackage{fancyhdr} % Per la capçalera
\usepackage{geometry}
\usepackage{graphicx} % Per importar logos (i altres gràfics)
\usepackage[colorlinks,linkcolor=black]{hyperref} % Per fer que l'index tingui hiperlinks en el pdf
\usepackage{indentfirst}
\usepackage{titlesec} % Seccions personalitzades

%% Títols
\newcommand{\modulnum}{1}
\newcommand{\modulnom}{Sistemes Informàtics}

\newcommand{\ufnum}{1}
\newcommand{\ufnom}{Instal·lació, Configuració \\ i Explotació del Sistema Informàtic}

\newcommand{\acttipus}{Pràctica 5.2}
\newcommand{\actnom}{Configuració del protocol de xarxa \\ en sistemes operatius lliures mitjançant la línia de comandes}

%% Entreliniats
\linespread{1.5}

%% Capçalera
\pagestyle{fancy}
\setlength{\headheight}{55pt}
\addtolength{\topmargin}{-25pt}
\fancyhead[L]{\includegraphics*[height=35pt]{provencana_bw.pdf}}
\fancyhead[R]{
	{\scshape\scriptsize Mòdul \modulnum: \modulnom}\\
	{\scshape\footnotesize \acttipus \space - \actnom}\\
	{\scshape\small Eina Coma Bages}}

\addtolength{\textheight}{2cm}

%% Secció
\titleformat{\enunciat}[runin]{\scshape}{\thesection.}{0.5em}{}


%% Comandes personalitzades
\newcommand{\mygraphic}[2][height=0.4\textheight]{\begin{center}
		\centering\includegraphics[#1]{#2}\par
\end{center}}

\newcounter{activitat}
\newcommand{\enunciat}[1]{\refstepcounter{activitat}
\vspace{0.5em} {\large\textit\theactivitat.} {\textit{#1}} \vspace{0.5em}}

\renewcommand{\contentsname}{Índex}

\begin{document}

\begin{titlepage}
	\centering
	\includegraphics*[width=0.15\textwidth]{provencana_color.pdf}
	\par\vspace{0.5cm}

	{\scshape\Large Institut Provençana \par}

	\vspace{1cm}
	
	{\itshape\Large Pràctica Optativa \par}
	{\bfseries\LARGE Instal·lació de Windows Server 2016 \par}
	
	\vspace{1cm}

	{\scshape\large Mòdul 1: \par}
	{\scshape\Large Sistemes Informàtics \par}

	\vspace{0.5cm}
	
	{\scshape\normalsize Unitat Formativa 1: \par}
	{\scshape\large Instal·lació, Configuració \\ i Explotació del Sistema Informàtic\par}

	\vfill
	{\Large\itshape Eina Coma Bages\par}
	
	\vfill
	Curs 2022/2023
\end{titlepage}
\newpage

\enunciat{Sol·licita la informació dels paràmetres de configuració de totes les targetes de xarxa instal·lades al teu sistema. Comprova si l'adaptador de la xarxa està en mode NAT.}

\mygraphic{imatges/1.png}

\enunciat{Sol·licita informació dels paràmetres de configuració de la targeta de xarxa Ethernet instal·lada al teu sistema}

\mygraphic{imatges/2.png}

\enunciat{Configura l'adaptador de la xarxa per estigui en mode pont (brigde). Torna a demanar la informació dels paràmetres de configuració de la targeta de xarxa Ethernet instal·lada al teu sistema. Han canviat els paràmetres de configuració de la targeta de xarxa? Per què?}

\mygraphic{imatges/3.png}

\enunciat{Desactiva i torna a activar la targeta de xarxa. Torna a demanar la informació dels paràmetres de configuració de la targeta de xarxa Ethernet. Han canviat els paràmetres de configuració de la targeta de xarxa? Per què?}

\mygraphic{imatges/4.png}

\newpage
\enunciat{Configura manualment els paràmetres de configuració de la targeta de xarxa Ethernet des de la línia de comandes. Assigna una nova IP, que serà igual a l'antiga, però sumant 100 al quart octet (sempre i quan, la inicial, no sigui superior a 154, ja que si és així, seria major a 254). La resta de paràmetres han de quedar coherents als de la nova IP. Comprova si els canvis han tingut efecte.}

\mygraphic{imatges/5.png}

\newpage
\enunciat{Afegeix una passarel·la (la adreça al router o gateway) als paràmetres de configuració de la targeta de xarxa Ethernet per tal de disposar de connexió externa (a Internet)}

\mygraphic{imatges/6.png}

\newpage
\enunciat{Consulta les taules d'encaminament del teu host i interpreta un parell de línies.}

\mygraphic{imatges/7.png}

\enunciat{Consulta les taules arp del teu host}

\mygraphic{imatges/8.png}

\enunciat{Accedeix a l'arxiu de configuració de la xarxa d'Ubuntu i configura manualment els paràmetres de configuració de la targeta de xarxa Ethernet, assignant una nova IP estàtica, obtinguda de sumar 1 al quart octet de la IP antiga. La resta de paràmetres han de quedar coherents als de la nova IP, i has de disposar de connexió externa (a Internet).}

El fitxer \texttt{/etc/network/interfaces} no existeix en instal·lacions modernes des que Ubuntu 17.10 va deprecar el paquet \texttt{ifupdown}.\footnote{\url{https://wiki.ubuntu.com/ArtfulAardvark/ReleaseNotes\#Network_configuration}}

\enunciat{Comprova a l'ordre ifconfig, si els canvis han tingut efecte. Si no és així, restaura el servei de xarxa i tornar-lo a provar.}

\enunciat{Comprova la connectivitat de la teva màquina amb la teva màquina amfitriona (la virtual haurà d'estar en mode pont) enviant-li 4 paquets de 64 bytes.}

\mygraphic{imatges/11.png}

\newpage
\enunciat{Fes un ping a la adreça broadcast.}

\mygraphic{imatges/12.png}

\enunciat{Esbrina tots els routers pels quals passen els paquets TCP/IP fins a arribar al servidor de Google.}

\mygraphic{imatges/13.png}

\enunciat{Accedeix a l'arxiu de configuració de la xarxa d'Ubuntu i fes que el servidor DHCP del centre configuri els paràmetres de la targeta de xarxa Ethernet. Comprova amb l'ordre \textbf{ifconfig}, si els canvis han tingut efecte. Si no és així, restaura el servei de xarxa i tornar-lo a provar.}

\enunciat{Quin és el nom del teu host? Canvia'l pel de “\textbf{ord4rtoctect\_IP}” on “\textbf{4rtoctect\_IP}” és el número corresponent al 4t octet de la IP del host.}

\mygraphic{imatges/15.png}

\enunciat{Configura com servidors DNS de la configuració TCP/IP, el servidors de noms de Google i Xtec.}

\newpage
\mygraphic{imatges/16.png}

\enunciat{Fes un ping a la IP al host del teu company i al nom del mateix host del teu company. Què succeeix? Fes que el ping que fas al nom del host del teu company sigui excitós.}

\mygraphic{imatges/17-1.png}
\mygraphic{imatges/17-2.png}
\mygraphic{imatges/17-3.png}

\newpage
\enunciat{Fes que al sol·licitar la pàgina web del Barça surti la pagina del R. Madrid, i a l'inrevés.}

\mygraphic{imatges/18-1.png}
\mygraphic{imatges/18-2.png}

\enunciat{Visualitza l'arxiu \textbf{/etc/nsswitch.conf} i indica l'''ordre amb la qual el “resolver” fa la resolució de noms.}

\enunciat{Què informació ens dona l'ordre \textbf{netstat}?}

\enunciat{Indica cóm s'accedeix a les eines de configuració de la xarxa en mode gràfic.}

\mygraphic{imatges/21-1.png}
\mygraphic{imatges/21-2.png}

\enunciat{Sol·licita la informació dels paràmetres de configuració de la targeta de xarxa inalàmbrica del teu sistema.}

\mygraphic{imatges/22.png}

\enunciat{Executa l'ordre \textbf{tracepath www.google.es}}

\mygraphic{imatges/23.png}

\newpage
\enunciat{Executa l'ordre \textbf{nmap -n -sP -PE 192.168.19.1,254}. Què fa?}

Envia a les IPs \texttt{192.168.19.1} i \texttt{192.168.19.254} una sol·licitud ICMP de tipus 8 (\texttt{-PE}), sense resoldre DNS (\texttt{-n}) ni escanejar-ne els ports (\texttt{-sP} àlies d'\texttt{-ns}, ja que el primer complia la mateixa funció en versions antigues).

\end{document}