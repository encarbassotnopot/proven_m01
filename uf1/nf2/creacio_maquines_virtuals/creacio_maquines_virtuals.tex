\documentclass[a4paper,12pt]{article}
\usepackage[utf8]{inputenc}

\usepackage{fancyhdr} % Per la capçalera
\usepackage{geometry}
\usepackage{graphicx} % Per importar logos (i altres gràfics)
\usepackage[colorlinks,linkcolor=black]{hyperref} % Per fer que l'index tingui hiperlinks en el pdf
\usepackage{parskip}
\usepackage[sc,small]{titlesec} % Seccions personalitzades

%% Entreliniats
\linespread{1.5}

%% Capçalera
\pagestyle{fancy}
\setlength{\headheight}{40pt}
\addtolength{\topmargin}{-20pt}
\fancyhead[L]{\includegraphics*[height=\headheight]{provencana_bw.pdf}}
\fancyhead[R]{
	{\scshape\scriptsize Mòdul 1: Sistemes Informàtics}\\
	{\scshape\footnotesize Pràctica Guiada - Creació de Màquines Virtuals}\\
	{\scshape\small Eina Coma Bages}}

\addtolength{\textheight}{2cm}

%% Comandes personalitzades
\newcommand{\mygraphic}[2][width=0.9\textwidth]{\begin{center}
		\centering\includegraphics[#1]{#2}\par
\end{center}}

\renewcommand{\contentsname}{Índex}

\begin{document}
\begin{titlepage}
	\centering
	\includegraphics*[width=0.15\textwidth]{provencana_color.pdf}
	\par\vspace{0.5cm}

	{\scshape\Large Institut Provençana \par}

	\vspace{1cm}
	
	{\itshape\Large Activitat \par}
	{\bfseries\LARGE Codificació de Caràcters \par}
	
	\vspace{1cm}

	{\scshape\large Mòdul 1: \par}
	{\scshape\Large Sistemes Informàtics \par}

	\vspace{0.5cm}
	
	{\scshape\normalsize Unitat Formativa 1: \par}
	{\scshape\large Instal·lació, Configuració \\ i Explotació del Sistema Informàtic\par}

	\vfill
	{\Large\itshape Eina Coma Bages\par}
	
	\vfill
	Curs 2022/2023
\end{titlepage}

\tableofcontents
\newpage

\section{Instal·lació de VirtualBox a la màquina amfitriona}

Comprovem que VirtualBox, les addicions i els mòduls de kernel ja estan instal·lats amb la següent ordre:
\texttt{pacman -Q virtualbox virtualbox-guest-iso virtualbox-host-modules-arch}.

Si no ho estiguessin ho podríem fer amb l'ordre \texttt{pacman -S} i els noms dels paquets.
\mygraphic{imatges/a0.png}

\newpage
\section{Instal·lació d'XP}
Creem una nova màquina virtual, l'anomenarem \texttt{xp-practica} i augmentarem a 512 MB dels 192 MB de RAM que VirtualBox suggereix per defecte.
\mygraphic{imatges/a1.png}
\mygraphic{imatges/a2.png}
\mygraphic{imatges/a3.png}

Creem un disc dur tipus VDI, allotjat dinàmicament, de 50 GB (en comptes dels 10 GB que surten per defecte).
\mygraphic{imatges/a4.png}
\mygraphic{imatges/a5.png}
\mygraphic{imatges/a6.png}
\mygraphic{imatges/a7.png}

Una vegada acabat el procés de creació, seleccionem la màquina que acabem de crear i anem a configuració, pestanya Emmagatzematge i seleccionem el disc, que està buit. Pitgem la icona de la dreta i triem l'ISO de Windows XP.
\mygraphic{imatges/a21.png}
\mygraphic{imatges/a22.png}
\mygraphic{imatges/a24.png}
\mygraphic{imatges/a25.png}

Ara ja podem iniciar la màquina i començar amb la instal·lació. Seleccionem l'espai sense particionar i el formatem amb NTFS i seguim els passos de l'instal·lador.
\mygraphic{imatges/b8.png}
\mygraphic{imatges/b9.png}
\mygraphic{imatges/b10.png}
\mygraphic[height=0.45\textheight]{imatges/b11.png}
\mygraphic[height=0.45\textheight]{imatges/b12.png}
\mygraphic[height=0.45\textheight]{imatges/b13.png}
\mygraphic[height=0.45\textheight]{imatges/b14.png}
\mygraphic[height=0.45\textheight]{imatges/b15.png}
\mygraphic[height=0.45\textheight]{imatges/b16.png}
\mygraphic[height=0.45\textheight]{imatges/b17.png}
\mygraphic[height=0.45\textheight]{imatges/b18.png}
\mygraphic[height=0.45\textheight]{imatges/b19.png}

Amb la instal·lació feta, ja podem treure el disc.
\mygraphic[height=0.45\textheight]{imatges/b28.png}
\mygraphic{imatges/b30.png}
\mygraphic{imatges/b29.png}

\subsection{Clonació de la màquina}
Fem clic dret a la nostra màquina i seleccionem clonar. Ens estan bé les opcions per defecte.
\mygraphic{imatges/c23.png}
\mygraphic[height=0.45\textheight]{imatges/c24.png}
\mygraphic{imatges/c25.png}
\mygraphic{imatges/c26.png}
\mygraphic{imatges/c27.png}

\subsection{Exportació}
Tornem a fer clic dret a la nostra màquina, però aquesta vegada seleccionarem exportar. També ens estan bé les opcions per defecte.
\mygraphic[height=0.45\textheight]{imatges/d30.png}
\mygraphic{imatges/d32.png}
\mygraphic{imatges/d33.png}

\newpage
\section{Instal·lació d'Ubuntu}
Canonical ja no ofereix els checksums MD5 de les versions més recents d'Ubuntu, només els SHA256, per tant farem servir aquest últim algoritme.
Descarreguem el fitxer \texttt{SHA256SUMS}, que conté les signatures per les ISOs de la versió 22.04, al mateix directori on prèviament ja hem descarregat l'ISO que volem validar.

Com que aquest arxiu conté els hashes de les versions d'escriptori i la de servidor, ens dona un error, ja que no pot trobar la segona imatge, però el que a nosaltres ens importa és que aconsegueix verificar la primera.
\mygraphic{imatges/e00.png}

El procediment per crear la màquina és el mateix que hem descrit abans amb Windows. Posarem de nom \texttt{ubuntu-practica}, 4 GB de ram i 50 de disc dur. 
\mygraphic{imatges/e01.png}
\mygraphic{imatges/e02.png}
\mygraphic{imatges/e03.png}
\mygraphic{imatges/e04.png}
\mygraphic{imatges/e05.png}
\mygraphic{imatges/e06.png}
\mygraphic{imatges/e07.png}
\mygraphic{imatges/e08.png}
\mygraphic{imatges/e09.png}
\mygraphic{imatges/e10.png}
\mygraphic{imatges/e11.png}

Ara ja ho tenim tot a punt per a començar la instal·lació.
\mygraphic{imatges/f00.png}
\mygraphic{imatges/f01.png}
\mygraphic[height=0.45\textheight]{imatges/f02.png}
\mygraphic[height=0.45\textheight]{imatges/f03.png}
\mygraphic[height=0.45\textheight]{imatges/f04.png}
\mygraphic[height=0.45\textheight]{imatges/f05.png}
\mygraphic[height=0.45\textheight]{imatges/f06.png}
\mygraphic[height=0.45\textheight]{imatges/f07.png}
\mygraphic[height=0.45\textheight]{imatges/f08.png}
\mygraphic[height=0.45\textheight]{imatges/f09.png}
\mygraphic[height=0.45\textheight]{imatges/f11.png}

\newpage
Una vegada hem reiniciat, procedim a instal·lar \texttt{gparted}, però no sense abans assegurar-nos que tenim l'última versió dels paquets ja instal·lats.

\texttt{apt update} per refrescar les llistes de paquets i \texttt{apt upgrade} per actualitzar els que tinguin disponibles noves versions. Finalment ja podem donar l'ordre d'\texttt{apt install gparted}.
\mygraphic[height=0.45\textheight]{imatges/f12.png}
\mygraphic[height=0.45\textheight]{imatges/f13.png}
\mygraphic[height=0.45\textheight]{imatges/f14.png}

\subsection{Snapshot}
A la barra de menú, seleccionem \texttt{Machine} i \texttt{Take Snapshot}. El nom per defecte ens està bé i premem \texttt{OK}
\mygraphic{imatges/f15.png}
\mygraphic{imatges/f17.png}
\mygraphic{imatges/f18.png}

\subsection{Exportació}
Com hem fet abans amb l'XP, clic dret sobre la màquina i exportar.
\mygraphic{imatges/f19.png}
\mygraphic{imatges/f20.png}
\mygraphic{imatges/f21.png}
\mygraphic{imatges/f22.png}

\newpage
\section{Guest Additions i carpeta compartida}
Creem la carpeta \texttt{host} al directori home del nostre usuari i a dins un fitxer de text, per ara buit, que anomenarem \texttt{shared.txt}.
\mygraphic{imatges/g1.png}

\subsection{Windows XP}
Inserim les Guest Additions. Quan obrim la unitat \texttt{D:} ja s'executa automàticament l'instal·lador, i seguim les instruccions. Una vegada acabada, reiniciem la màquina.
\mygraphic[height=0.45\textheight]{imatges/g2.png}
\mygraphic[height=0.45\textheight]{imatges/g3.png}
\mygraphic[height=0.45\textheight]{imatges/g4.png}
\mygraphic[height=0.45\textheight]{imatges/g5.png}
\mygraphic[height=0.45\textheight]{imatges/g6.png}
\mygraphic[height=0.45\textheight]{imatges/g7.png}
\mygraphic[height=0.45\textheight]{imatges/g8.png}
Una vegada reiniciada, podem accedir a la carpeta que hem creat anteriorment, que apareixerà com una unitat de xarxa compartida amb lletra \texttt{Z:}. I a dins, el fitxer shared.txt, on escriurem "Modificat des de Windows XP".
També podrem configurar el porta-retalls i l'arrossegament de fitxers bidireccional.
\mygraphic{imatges/g9.png}
\mygraphic[height=0.45\textheight]{imatges/g10.png}
\mygraphic[height=0.45\textheight]{imatges/g11.png}
\mygraphic[height=0.45\textheight]{imatges/g12.png}

\subsection{Ubuntu}
Inserim també les Guest Additions. En aquest cas, no se'ns han executat automàticament, per tant ho farem nosaltres. \texttt{./autorun.sh} per executar l'script que no ho ha automàticament. Posem la nostra contrasenya i esperem que acabi la instal·lació. Una vegada finalitzada reiniciem.
\mygraphic[height=0.45\textheight]{imatges/h1.png}
\mygraphic[height=0.45\textheight]{imatges/h2.png}
\mygraphic[height=0.45\textheight]{imatges/h3.png}
\mygraphic[height=0.45\textheight]{imatges/h4.png}
Aprofitem aquest impàs per compartir la carpeta.
\mygraphic[height=0.45\textheight]{imatges/h5.png}
Ens trobarem que \texttt{root} i els membres del grup \texttt{vboxsf} tenen tots els permisos d'aquesta carpeta, però la resta d'usuaris no en té cap. Ho solucionarem afegint el nostre usuari al grup, amb  l'ordre \texttt{usermod -aG vboxsf eina}.
\mygraphic[height=0.44\textheight]{imatges/h6.png}
\mygraphic[height=0.44\textheight]{imatges/h7.png}
Ara que ja hi tenim accés, podem editar el fitxer \texttt{shared.txt}, on trobarem la línia que hem afegit des de Windows. Escriurem "Modificat des d'Ubuntu" i el desarem.
Finalment, també activarem el porta-retalls i l'arrossegament bidireccional de fitxers.
\mygraphic[height=0.45\textheight]{imatges/h8.png}
\mygraphic[height=0.45\textheight]{imatges/h10.png}
\mygraphic[height=0.45\textheight]{imatges/h11.png}

\newpage
\section{Canvi d'UUID}
La manera de clonar un disc i que tingui un UUID diferent de l'original, és amb l'ordre \texttt{vboxmanage clonemedium}. Aquesta pren com a argument el nom del disc que volem clonar i el nom que volem donar al clon.
\mygraphic{imatges/i1.png}

\end{document}