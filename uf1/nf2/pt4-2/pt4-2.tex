\documentclass[a4paper,12pt]{article}
\usepackage[utf8]{inputenc}

\usepackage[spanish]{babel} % Patrons de trencament de paraula
\usepackage{fancyhdr} % Per la capçalera
\usepackage{geometry}
\usepackage{graphicx} % Per importar logos (i altres gràfics)
\usepackage[colorlinks,linkcolor=black]{hyperref} % Per fer que l'index tingui hiperlinks en el pdf
\usepackage{indentfirst}
\usepackage[sc,small]{titlesec} % Seccions personalitzades

%% Títols
\newcommand{\modulnum}{1}
\newcommand{\modulnom}{Sistemes Informàtics}

\newcommand{\ufnum}{1}
\newcommand{\ufnom}{Instal·lació, Configuració \\ i Explotació del Sistema Informàtic}

\newcommand{\acttipus}{Pràctica 4.2}
\newcommand{\actnom}{Gestió d'usuaris i grups a Windows}

%% Entreliniats
\linespread{1.5}

%% Capçalera
\pagestyle{fancy}
\setlength{\headheight}{40pt}
\addtolength{\topmargin}{-20pt}
\fancyhead[L]{\includegraphics*[height=35pt]{provencana_bw.pdf}}
\fancyhead[R]{
	{\scshape\scriptsize Mòdul \modulnum: \modulnom}\\
	{\scshape\footnotesize \acttipus \space  - \actnom}\\
	{\scshape\small Eina Coma Bages}}

\addtolength{\textheight}{2cm}

%% Comandes personalitzades
\newcommand{\mygraphic}[2][height=0.4\textheight]{\begin{center}
		\centering\includegraphics[#1]{#2}\par
\end{center}}

\renewcommand{\contentsname}{Índex}

\begin{document}
\begin{titlepage}
	\centering
	\includegraphics*[width=0.15\textwidth]{provencana_color.pdf}
	\par\vspace{0.5cm}

	{\scshape\Large Institut Provençana \par}

	\vspace{1cm}
	
	{\itshape\Large Activitat \par}
	{\bfseries\LARGE Codificació de Caràcters \par}
	
	\vspace{1cm}

	{\scshape\large Mòdul 1: \par}
	{\scshape\Large Sistemes Informàtics \par}

	\vspace{0.5cm}
	
	{\scshape\normalsize Unitat Formativa 1: \par}
	{\scshape\large Instal·lació, Configuració \\ i Explotació del Sistema Informàtic\par}

	\vfill
	{\Large\itshape Eina Coma Bages\par}
	
	\vfill
	Curs 2022/2023
\end{titlepage}

\newpage
\textsc{Crea dos usuaris, un anomenat usuari1cognom1 i altre anomenat usuari2cognom1, ací queda demostrat com s'utilitzen les tres caselles de l’usuari.}


\begin{itemize}
	\item \textsc{El usuari no pot canviar la contrasenya (usuari1cognom1)}
	\mygraphic{imatges/1.png}
	\newpage
	\item \textsc{Compte deshabilitat (crea usuari3cognom1)}
	\mygraphic{imatges/2.png}
	\item \textsc{La contrasenya mai caduca (usuari2cognom1)}
	\mygraphic{imatges/3.png}
\end{itemize}

\newpage
\textsc{Quines condicions ha d'acomplir la contrasenya?}

No pot tenir més de 14 caràcters.
\mygraphic{imatges/4.png}

\newpage
\textsc{Creació de dos usuaris més amb les següents característiques i els següents grups}

\begin{itemize}
	\item \textsc{El usuari1 haurà de ser del tipus convidat(grup1)}
	\mygraphic{imatges/5.png}
	\newpage
	\item \textsc{El usuari2 haurà de ser del tipus usuari avançat(grup1)}
	\mygraphic{imatges/6.png}
	\item \textsc{El usuari3 haurà de ser un duplicador(grup2)}
	\mygraphic{imatges/7.png}
	\item \textsc{El usuari4 haurà de ser administrador(grup2)}
	\mygraphic{imatges/8.png}
\end{itemize}

\textsc{Crea una carpeta des de usuari2cognom1 amb un document de text dins sobre l'escriptori}
\mygraphic{imatges/9.png}

\newpage
\textsc{Intenta'l modificar des de usuari1cognom1. Pots? Mostra el que apareix per pantalla}

No, no es pot.
\mygraphic{imatges/10.png}

\textsc{Intenta'l modificar des de usuari2cognom1. Pots? Mostra el que apareix per pantalla}
\mygraphic{imatges/11.png}


\end{document}