\documentclass[a4paper,12pt]{article}
\usepackage[utf8]{inputenc}

\usepackage[catalan]{babel} % Patrons de trencament de paraula
\usepackage{fancyhdr} % Per la capçalera
\usepackage{geometry}
\usepackage{graphicx} % Per importar logos (i altres gràfics)
\usepackage[colorlinks,linkcolor=black]{hyperref} % Per fer que l'index tingui hiperlinks en el pdf
\usepackage{indentfirst}
\usepackage[sc,small]{titlesec} % Seccions personalitzades

%% Títols
\newcommand{\modulnum}{1}
\newcommand{\modulnom}{Sistemes Informàtics}

\newcommand{\ufnum}{1}
\newcommand{\ufnom}{Instal·lació, Configuració \\ i Explotació del Sistema Informàtic}

\newcommand{\acttipus}{Pràctica 5.6}
\newcommand{\actnom}{Protocols de capa d'aplicació}

%% Entreliniats
\linespread{1.5}

%% Capçalera
\pagestyle{fancy}
\setlength{\headheight}{40pt}
\addtolength{\topmargin}{-20pt}
\fancyhead[L]{\includegraphics*[height=35pt]{provencana_bw.pdf}}
\fancyhead[R]{
	{\scshape\scriptsize Mòdul \modulnum: \modulnom}\\
	{\scshape\footnotesize \acttipus \space  - \actnom}\\
	{\scshape\small Eina Coma Bages}}

\addtolength{\textheight}{2cm}

%% Comandes personalitzades
\newcommand{\mygraphic}[2][height=0.4\textheight]{\begin{center}
		\centering\includegraphics[#1]{#2}\par
\end{center}}

\renewcommand{\contentsname}{Índex}

\begin{document}
\begin{titlepage}
	\centering
	\includegraphics*[width=0.15\textwidth]{provencana_color.pdf}
	\par\vspace{0.5cm}

	{\scshape\Large Institut Provençana \par}

	\vspace{1cm}
	
	{\itshape\Large Pràctica Optativa \par}
	{\bfseries\LARGE Instal·lació de Windows Server 2016 \par}
	
	\vspace{1cm}

	{\scshape\large Mòdul 1: \par}
	{\scshape\Large Sistemes Informàtics \par}

	\vspace{0.5cm}
	
	{\scshape\normalsize Unitat Formativa 1: \par}
	{\scshape\large Instal·lació, Configuració \\ i Explotació del Sistema Informàtic\par}

	\vfill
	{\Large\itshape Eina Coma Bages\par}
	
	\vfill
	Curs 2022/2023
\end{titlepage}

\tableofcontents
\newpage

\section{Realitza les següents configuracions}
Configura un servidor Web. Pots copiar el codi font d’un altre lloc web.
\mygraphic{imatges/01.png}

\textsc{Configura un servidor DNS per tal que resolgui noms URL’s a IP’s.}
\mygraphic{imatges/02.png}

\newpage
\textsc{Configura un servidor DHCP, per tal d’assignar IP’s de manera dinàmica en el rang 192.168.1.10 fins 192.168.1.15}
\mygraphic{imatges/03.png}

\textsc{Configura un client amb IP fixa: 192.168.1.5}
\mygraphic{imatges/04.png}

\newpage
\textsc{Configura un client amb IP dinàmica assignada pel servidor DHCP}
\mygraphic{imatges/05.png}

\newpage
\section{Explica el resultat de realitzar les següents comprovacions}
\textsc{Connectivitat entre tots el dispositius de la xarxa}

Tots els dispositius poden arribar als altres.
\mygraphic[width=\textwidth]{imatges/06.png}

\textsc{Fes, des de el client, ping al \textbf{www.example.com} i \textbf{dhcp.example.com}}
\mygraphic{imatges/07.png}

\newpage
\textsc{Anul·la el servei DNS i torna a fer pings a \textbf{www.example.com} i \textbf{dhcp.example.com}. Ara fes  ping’s a les seves IP’s.}
\mygraphic{imatges/08.png}
Sense DNS no pot resoldre els hostnames, però amb les adreces IP es pot fer ping als servidors.
\mygraphic{imatges/09.png}

\newpage
\textsc{Torna a activar el servei DNS i des de el navegador web (en Desktop/web browser) introduïu l’adreça \textbf{www.example.com} i \textbf{dhcp.example.com}}
\mygraphic{imatges/10.png}

El navegador no mostra la pàgina, però la pantalla fa un flaix blanc (que no fa amb el servidor dns apagat), com si intentés carregar alguna cosa.
\mygraphic{imatges/11.png}
\mygraphic{imatges/12.png}

\end{document}