\documentclass[a4paper,12pt]{article}
\usepackage[utf8]{inputenc}

\usepackage[catalan]{babel} % Patrons de trencament de paraula
\usepackage{fancyhdr} % Per la capçalera
\usepackage{geometry}
\usepackage{graphicx} % Per importar logos (i altres gràfics)
\usepackage[colorlinks,linkcolor=black]{hyperref} % Per fer que l'index tingui hiperlinks en el pdf
\usepackage{indentfirst}
\usepackage[sc,small]{titlesec} % Seccions personalitzades
% per donar format al codi
\usepackage{xcolor}
\usepackage{listings}
\usepackage{listingsutf8}

%% Títols
\newcommand{\modulnum}{1}
\newcommand{\modulnom}{Sistemes Informàtics}

\newcommand{\ufnum}{2}
\newcommand{\ufnom}{Gestió de la informació i de recursos en una xarxa.}

\newcommand{\acttipus}{Pràctica}
\newcommand{\actnom}{Scripting en bash}

\newcommand*{\autor}{Eina Coma Bages}

%% Entreliniats
\linespread{1.5}

%% Capçalera
\pagestyle{fancy}
\setlength{\headheight}{40pt}
\addtolength{\topmargin}{-20pt}
\fancyhead[L]{\includegraphics*[height=35pt]{provencana_bw.pdf}}
\fancyhead[R]{
	{\scshape\scriptsize Mòdul \modulnum: \modulnom}\\
	{\scshape\footnotesize \acttipus \space  - \actnom}\\
	{\scshape\small\autor}}

\addtolength{\textheight}{2cm}

%% Comandes personalitzades
\newcommand{\mygraphic}[2][width=\textwidth]{\begin{center}
		\centering\includegraphics[#1]{#2}\par
\end{center}}

\renewcommand{\contentsname}{Índex}

% Format codi

\definecolor{codegreen}{rgb}{0,0.6,0}
\definecolor{codegray}{rgb}{0.5,0.5,0.5}
\definecolor{codepurple}{rgb}{0.58,0,0.82}
\definecolor{backcolour}{rgb}{0.95,0.95,0.92}

\lstdefinestyle{mystyle} {
	backgroundcolor=\color{backcolour},
	commentstyle=\color{codegreen},
	keywordstyle=\color{magenta},
	numberstyle=\tiny\color{codegray},
	stringstyle=\color{codepurple},
	basicstyle=\ttfamily\footnotesize,
	breakatwhitespace=true,
	breaklines=true,
	captionpos=b,
	keepspaces=true,
	numbers=left,
	numbersep=5pt,
	showspaces=false,
	showstringspaces=false,
	showtabs=false,
	tabsize=4
}

\lstset{style=mystyle}

\begin{document}
\begin{titlepage}
	\centering
	\includegraphics*[width=0.15\textwidth]{provencana_color.pdf}
	\par\vspace{0.5cm}

	{\scshape\Large Institut Provençana \par}

	\vspace{1cm}
	
	{\itshape\Large Pràctica Optativa \par}
	{\bfseries\LARGE Instal·lació de Windows Server 2016 \par}
	
	\vspace{1cm}

	{\scshape\large Mòdul 1: \par}
	{\scshape\Large Sistemes Informàtics \par}

	\vspace{0.5cm}
	
	{\scshape\normalsize Unitat Formativa 1: \par}
	{\scshape\large Instal·lació, Configuració \\ i Explotació del Sistema Informàtic\par}

	\vfill
	{\Large\itshape Eina Coma Bages\par}
	
	\vfill
	Curs 2022/2023
\end{titlepage}

\newpage
\textsc{1. Crea l'arxiu "sinfin" amb el següent contingut:}
\begin{lstlisting}[language=bash]
while true
do
echo "NO jugaré a los videojuegos en clase y sólo utilizaré el ordenador con fines académicos"
sleep 3
done
\end{lstlisting}

\textsc{Dona permisos d'execució al programa i executa'l. Este programa genera un bucle infinit.
Executa el programa en primer pla. ¿Pots llistar el teu directori personal?}

\mygraphic{imatges/1.png}
No, no puc llistar-lo Amb el programa en primer pla.

\textsc{Executa el programa en segon pla. ¿Pots llistar el tu directori personal? Como interrompre'l? Como passar-lo a segon pla? }

Amb el programa en segon pla si que el podem llistar. Per interrompre el programa en primer pla, premem \texttt{Ctrl+C}, i per fer-ho quan està en segon pla executem \texttt{kill \%}. Per passar-lo de primer a segon pla, s'ha de premer \texttt{Ctrl+Z}

\newpage
\textsc{2. Fes un script que crea una carpeta principal anomenada “Qualitat” y crea 3 subcarpetes anomenades Q1,Q2 i Q3.}
\lstinputlisting[language=Bash]{scripts/2.sh}
\mygraphic{imatges/2.png}

\newpage
\textsc{3. Script que crea moltes carpetes en una. Volem un script que cree aquestes carpetes:
\texttt{cicles=\{dam\{c1,c2\}, daw\{c1,c2\}, smx\{c1,c2\}, asix\{c1,c2\}, dawbio\{c1,c2\}\}}}
\lstinputlisting[language=Bash]{scripts/3.sh}
\mygraphic{imatges/3.png}

\newpage
\textsc{4. Script què, donat el nom d’una persona, mostra la salutació: \linebreak
\texttt{Hola <nom>! \linebreak
Avui és el dia <date>.}}
\lstinputlisting[language=Bash]{scripts/4.sh}
\mygraphic{imatges/4.png}

\newpage
\textsc{5. Script que donat un número retorne la seua taula multiplicar.}
\lstinputlisting[language=Bash]{scripts/5.sh}
\mygraphic{imatges/5.png}

\newpage
\textsc{6. Fes els exercicis 2 i 3, però amb la diferència que el primer paràmetre sigui el nom de la carpeta. Que surti un missatge si la carpeta ja existeix.}
\lstinputlisting[language=Bash]{scripts/6-1.sh}
\lstinputlisting[language=Bash]{scripts/6-2.sh}
\mygraphic{imatges/6.png}

\end{document}