\documentclass[a4paper,12pt]{article}
\usepackage[utf8]{inputenc}

\usepackage[catalan]{babel} % Patrons de trencament de paraula
\usepackage{fancyhdr} % Per la capçalera
\usepackage{geometry}
\usepackage{graphicx} % Per importar logos (i altres gràfics)
\usepackage[colorlinks,linkcolor=black]{hyperref} % Per fer que l'index tingui hiperlinks en el pdf
\usepackage{indentfirst}
\usepackage[sc,small]{titlesec} % Seccions personalitzades

%% Títols
\newcommand{\modulnum}{1}
\newcommand{\modulnom}{Sistemes Informàtics}

\newcommand{\ufnum}{2}
\newcommand{\ufnom}{Gestió de la informació i de recursos en una xarxa.}

\newcommand{\acttipus}{Pràctica}
\newcommand{\actnom}{Scripting en bash}

\newcommand*{\autor}{Eina Coma Bages}

%% Entreliniats
\linespread{1.5}

%% Capçalera
\pagestyle{fancy}
\setlength{\headheight}{40pt}
\addtolength{\topmargin}{-20pt}
\fancyhead[L]{\includegraphics*[height=35pt]{provencana_bw.pdf}}
\fancyhead[R]{
	{\scshape\scriptsize Mòdul \modulnum: \modulnom}\\
	{\scshape\footnotesize \acttipus \space  - \actnom}\\
	{\scshape\small\autor}}

\addtolength{\textheight}{2cm}

%% Comandes personalitzades
\newcommand{\mygraphic}[2][width=\textwidth]{\begin{center}
		\centering\includegraphics[#1]{#2}\par
\end{center}}

\renewcommand{\contentsname}{Índex}

\begin{document}
\begin{titlepage}
	\centering
	\includegraphics*[width=0.15\textwidth]{provencana_color.pdf}
	\par\vspace{0.5cm}

	{\scshape\Large Institut Provençana \par}

	\vspace{1cm}
	
	{\itshape\Large Pràctica Optativa \par}
	{\bfseries\LARGE Instal·lació de Windows Server 2016 \par}
	
	\vspace{1cm}

	{\scshape\large Mòdul 1: \par}
	{\scshape\Large Sistemes Informàtics \par}

	\vspace{0.5cm}
	
	{\scshape\normalsize Unitat Formativa 1: \par}
	{\scshape\large Instal·lació, Configuració \\ i Explotació del Sistema Informàtic\par}

	\vfill
	{\Large\itshape Eina Coma Bages\par}
	
	\vfill
	Curs 2022/2023
\end{titlepage}


\newpage
\textsc{1. Visualitza les particions del disc dur de la maquina virtual on s'ha instal\textperiodcentered lat Ubuntu, des d'un entorn gràfic i des de la línia de comandes. Visualitza les particions que el sistema ha muntat automàticament.}

\newpage
\textsc{2. Analitza l'espai dels dispositius presents (devices). Llistar l'espai total i el disponible de cada partició muntada, tot mostrant el sistema de fitxers que utilitza. Cal mostrar la informació en unitas K, M o G. Llistar només els sistemes de fitxers ext3. Llistar només els sistemes de fitxers virtuals. Quins són?}

\newpage
\textsc{3. Llista l'ocupació de l'espai de disc. Llistar (K, M o G) l'ocupació de disc de /boot. Llistar l'ocupació de disc de /etc resumint a un nivell de profunditat.}

\newpage
\textsc{4. Visualitza la informació de l'ordre “free” i l'ordre “stat” pel directori /etc i per l'arxiu /etc/fstab.}

\newpage
\textsc{5. Afegeix a la teva maquina virtual 3 disc durs sata de 10, 15 i 20 GB respectivament. Fes una captura de pantalla dels dispositius d' emmagatzematze i de les taules de particions de cadascun dels discs instal·lats.}

\newpage
\textsc{6. Utilitza la línia de comandes per fer dues particions primàries contigües en el primer disc sata (/dev/sdb). La primera partició de 3GB de capacitat i la segona de 4GB.}

\newpage
\textsc{7. Assigna l'etiqueta “particio\_linux” a la primera partició i l'etiqueta “particio\_windows” a la segona partició. Formata amb sistema d'arxius ext4 la primera partició (particio\_linux) i amb sistema d'arxius ntfs la segona partició (particio\_windows).}

\newpage
\textsc{8. Visualitza el contingut del directori /mnt. Munta la primera partició en aquest directori (/mnt). Visualitza novament el contingut del directori /mnt i afegeix un arxiu de text (prueba.txt). Si no pot fer-ho, afegeix-li els permisos de lectura, escriptura i execució per a tots el usuaris (sudo chmod 777 /mnt). Per últim, desmunta la partició.}

\newpage
\textsc{9. Comprova la integritat dels sistemes d'arxius de les dues particions del disc hdb, i del disc del sistema (hda).}

\newpage
\textsc{10. Visualitza el contingut del directori /mnt. Munta novament la primera partició en el directori (/mnt), i torna-la a muntar en calent, però en mode de “només lectura” (ro). Intenta crear un directori anomenat carpeta amb l'ordre mkdir, i llegeix l'arxiu Prueba amb l'ordre cat.}

\newpage
\textsc{11. Visualitza l'espai en els dispositius presents (devices).Llistar l'espai total i el disponible de cada partició muntada, tot mostrant el sistema de fitxers que utilitza.}

\newpage
\textsc{12. Llista els “swaps” actius. Crea una tercera partició primària en el disc sdb, de 1GB de capacitat, destinada a memòria virtual. Activa i desactiva el swap manualment.}

\end{document}